\documentclass[a4paper, 11pt]{article}
\usepackage[utf8]{inputenc}
\usepackage[T1]{fontenc}
\usepackage{lmodern}
\usepackage{graphicx}
\usepackage[french]{babel}
\usepackage{subfigure}
\usepackage{amsmath, amsfonts, amssymb}

\setlength{\textheight}{24.2cm}
\setlength{\textwidth}{16.0cm}
\setlength{\oddsidemargin}{0.0cm}
\setlength{\evensidemargin}{0.0cm}
\setlength{\topmargin}{-1cm}


\begin{document}


\title{Avancement Stage TDF}
\maketitle

\section{Mercredi 22/01}

Première séance, il y a eu une prise de contact entre Emile Baptiste et Hoël afin de jeter les bases du stage, présentations, présentation du projet, nous avons également légèrement survolé les procédés que nous allons mettre en oeuvre durant cette période.
   Puis nous avons travaillé sur l'outil Matlab afin de déterminer certaines caractéristiques des données cf stage1.m, dans le but de voir ce qui était le plus pertinent, et de réfléchir si nous aurions à demander d'autres informations à Tdf, et le cas échéant lesquelles.

\section{Vendredi 24/01}

Prise de contact de Laure quivy avec les étudiants, les premières estimation de dates ont été jetées, un code grossier devrait être fourni pour le mois de Mai (TDF a une réunion concernant les tarifs, ils aimeraient présenter ça à ce moment là), puis si on peut l'améliorer, continuer pour ne fournir une version plus performante en Juin ou Juillet.
Mise en place du Git hub en commun, vision ou révision de Matlab, méthodes d'interpolation, et de regression pour les deux étudiants.

\end{document}