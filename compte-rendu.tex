\documentclass[a4paper, 11pt]{article}
\usepackage[utf8]{inputenc}
\usepackage[T1]{fontenc}
\usepackage{lmodern}
\usepackage{graphicx}
\usepackage[french]{babel}
\usepackage{subfigure}
\usepackage{amsmath, amsfonts, amssymb}

\setlength{\textheight}{24.2cm}
\setlength{\textwidth}{16.0cm}
\setlength{\oddsidemargin}{0.0cm}
\setlength{\evensidemargin}{0.0cm}
\setlength{\topmargin}{-1cm}


\begin{document}


\title{Avancement Stage TDF}
\maketitle


\section{Mercredi 22/01}

	Première séance, il y a eu une prise de contact entre Emile Baptiste et Hoël afin de jeter les bases du stage, présentations, présentation du projet, nous avons également légèrement survolé les procédés que nous allons mettre en oeuvre durant cette période.
   Puis nous avons travaillé sur l'outil Matlab afin de déterminer certaines caractéristiques des données cf stage1.m, dans le but de voir ce qui était le plus pertinent, et de réfléchir si nous aurions à demander d'autres informations à Tdf, et le cas échéant lesquelles.


\section{Vendredi 24/01}

	Prise de contact de Laure quivy avec les étudiants, les premières estimation de dates ont été jetées, un code grossier devrait être fourni pour le mois de Mai (TDF a une réunion concernant les tarifs, ils aimeraient présenter ça à ce moment là), puis si on peut l'améliorer, continuer pour ne fournir une version plus performante en Juin ou Juillet.
	Mise en place du Git hub en commun, vision ou révision de Matlab, méthodes d'interpolation, et de regression pour les deux étudiants.


\section{Mercredi 29/01}

\subsection{R�union � TDF}
	Aujourd'hui nous avons eu la r�union avec Olivier Marzouk de TDF, pour lui pr�senter les premi�res analyses de donn�es.
	Apparemment la donn�e qui sensiblement pourrait nous manquer porte sur la date de construction des diff�rents sites, il faut donc voir par la suite si elle explique les points �tranges. Nous n'aurons pas acc�s au code du logiciel de l'ARCEP, mais une partie du calcul est disponible en ligne (� voir sur le lien suivant que nous a fourni Olivier: https://docs.google.com/open?id=0B4dfKcelvACpNFJIdDN2SEFXN00).
	Le premier nombre de panneau serait � priori inutile, c'est plut�t le nombre maximum de panneau (donc utiliser X(:,2) plut�t que X(:,1)).


\subsection{Hors r�union}
	Il reste � v�rifier si l'apparence de droites des graphes scatter(X(:,3),Y(:,1)) exhib�es par Baptiste restent valables en faisant varier les valeurs de X(:,2), X(:,4) et X(:,5). Si c'est le cas, il ne reste plus qu'� trouver ce qui fait varier le coefficient directeur de la droite, et on obtient une relation affine entre le prix et les crit�res... ce qui parait trop simple.
	Les valeurs �tranges ont �t� envoy�es � Olivier afin qu'il v�rifie qu'elles ne sont pas fausses (l'erreur est humaine).
	Ho�l a exhib� les sites qui apparaissent sur les graphes �tranges (du moins ceux qui sautent au yeux), c'est-�-dire ceux qui ont des caract�ristiques proches d'autres sites mais qui pr�sentent avec eux un �cart de prix d'au moins 25%. Il y en a une dizaine.
	Baptiste a �crit un script permettant d'exhiber toutes les configs qui ont les m�mes caract�ristiques mais pas le m�me prix, il y a 391 groupes comprenant chacun au moins 2 configs qui correspondent � cette description. Cel� confirme qu'il nous faut d'autres crit�res. 
	

\end{document}
